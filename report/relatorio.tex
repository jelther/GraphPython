\documentclass[10pt,a4paper]{report}


\usepackage[utf8]{inputenc}
\usepackage[portuguese]{babel}
\usepackage[T1]{fontenc}
\usepackage{amsmath}
\usepackage{amsfonts}
\usepackage{amssymb}
\usepackage{graphicx}
\usepackage{chngcntr}
\usepackage{fancyhdr}
\usepackage{caption}
\usepackage{indentfirst}
\usepackage[]{tocbibind}
\usepackage{listings}
\usepackage{minted}
\usepackage{lmodern}

\usepackage[toc]{appendix}

% hyperref nao esta funcionando com o appendix
% \usepackage[]{hyperref}
\usepackage[]{url}

\usepackage{titlesec}

\usepackage{pgffor}

% quebra de pagina a cada secao
%\newcommand{\sectionbreak}{\clearpage}
%\newcommand{\subsectionbreak}{\clearpage}


%========================================
%Refaz o estilo dos cabecalhos e rodape
%========================================
\fancypagestyle{plain}{
	\fancyhf{}
	\fancyhead[LE,RO]{ \footnotesize \today}
	\fancyhead[RE,LO]{ \footnotesize Algoritmos de Prim, Kruskal, Dijkstra e Ford-Moore-Bellman}
	\fancyfoot[CE,CO]{\leftmark}
	\fancyfoot[LE,RO]{\thepage}
}

\pagestyle{plain}

\renewcommand{\headrulewidth}{1pt}
\renewcommand{\footrulewidth}{1pt}
%========================================


%========================================
% Traduz a forca o nome dos apendices
%========================================
\renewcommand{\appendixtocname}{Apêndices}
%========================================

\begin{document}


%========================================
%Capa do trabalho esta em outro arquivo
%========================================
\begin{titlepage}
	\begin{center}

	\begin{minipage}{5in}
	  \centering
	  \raisebox{-0.5\height}{\includegraphics[height=1.25in]{logo-unicamp}}
	  \hspace*{.2in}
	  \raisebox{-0.5\height}{\includegraphics[height=1in]{logo-feec}}
	\end{minipage}
	~\\[1cm]

	\textsc{\LARGE Universidade Estadual de Campinas}\\[0.3cm]
	\textsc{\large Faculdade de Engenharia Elétrica e Computação}\\[1.5cm]
	\textsc{\Large Trabalho Computacional}\\[0.5cm]
	\textsc{\Large Teoria dos Grafos}\\[0.5cm]

	\hrule~\\[0.01cm]
	{ \huge \bfseries Algoritmos de Prim, Kruskal, Dijkstra e Ford-Moore-Bellman\\[0.4cm] }
	\hrule

	~\\[1cm]
	% Author and supervisor
	\noindent
	\begin{minipage}[t]{0.5\textwidth}
		\begin{flushleft} \large
			\emph{Autor:}\\
			Jelther Oliveira Gonçalves (097254)
			\\
			~\\
			\emph{Otimização Linear (IA881)}\\			
		\end{flushleft}
	\end{minipage}%
	\begin{minipage}[t]{0.5\textwidth}
		\begin{flushright} \large
			\emph{Professor:} \\
			Dr. Ricardo C. L. F. Oliveira
			\\
			~\\
			~\\
			\emph{2º Semestre de 2016}\\
		\end{flushright}
	\end{minipage}

	\vfill

	% Bottom of the page
	{\large \today}
	\end{center}
\end{titlepage}
%========================================



\abstract{

Neste presente trabalho foram aplicados os algoritmos de Prim,Kruskal,Dijkstra e Ford-Moore-Bellman em 2 grafos : Rede Óptica Italiana e Rede Rodoviária dos EUA.

A linguagem de programação utilizada foi o \textit{Python 2.7} juntamente com o pacote \textit{Networkx} como estrutura para os grafos e para efeito de comparação com os algoritmos já existentes neste pacote.

Para os algoritmos de Prim e Kruskal, foram gerados arquivos texto com os dados da árvore geradora mínima (suas arestas e nós) assim como o número de iterações, custo da árvore e o tempo computacional gasto.

Já para os algoritmos de Dijkstra e Ford-Moore-Bellman, foram apresentados os caminhos mínimos gerados para cada nó-fim solicitado, bem como o tamanho do caminho mínimo, número de iterações e relaxações aplicadas e o nó-anterior ao nó-fim.

}


\newpage

\tableofcontents

\newpage

\chapter{A linguagem Python}
A linguagem Python, conforme visto em \cite{python} e \cite{pythonwikipedia} é uma linguagem de alto nivel, interpretada, de script, multiplataforma, orientada a objetos, funcional,de tipagem dinâmica e forte. Foi criada em 1991 e é amplamente utilizada mundialmente.

Apesar de facilitar o desenvolvimento, em relação as linguagens compiladas existe uma penalidade na performance dos scripts criados \cite{pythonperformance}. Entretanto, utilizando de pacotes já desenvolvidos sobre rotinas compiladas,a perda de performance é menor.


\chapter{O pacote Networkx}
O pacote Networkx \cite{networkx} é utilizado para criação,manipulação e estudo de grafos e redes.

O pacote já possui vários métodos implementados \cite{networkxdocumentation} mas neste trabalho nos limitamos a utilizar somente a estrutura dos grafos e seus métodos para acessar os nós e arestas.



\chapter{Set-up do computador utilizado}
O computador em que estes algoritmos foram rodados possui a seguinte configuração:

\begin{itemize}  
\item Processador AMD FX-4300, Black Edition, Cache 8Mb, 3.8GHz, AM3+ FD4300WMHKBOX
\item 8 GB RAM Kingston 1333Mhz DDR3 CL9 - KVR13N9S8/4
\item Sistema Operacional Microsoft Windows 10 (build 14393), 64-bit
\end{itemize}


\chapter{Algoritmo de Prim}
\section{Resumo}
Nesta implementação o conjunto franja é construído a cada iteração. Saliento que este não é o melhor caminho em termos de performance, como pode ser visto com a comparação do tempo de execução com o algoritmo do pacote Networkx.

\section{Rede Italiana}
\inputminted[fontsize=\small,linenos,tabsize=2,breaklines]{text}{../output/prim_rede_italiana.txt}


\section{Rede USA}
\inputminted[fontsize=\small,linenos,tabsize=2,breaklines]{text}{../output/prim_rede_usa.txt}


\chapter{Algoritmo de Kruskal}
\section{Resumo}
Nesta implementação o a busca pelo ciclo no algoritmo é feita utilizando o DFS (Depth-first search) \cite{dfsalgorithm}. Em termos de performance, utilizar o Disjoint-Set Data Structure \cite{disjointsetalgorithm} tem uma performance melhor mas devido a problemas na implementação ele não foi utilizado.

\section{Rede Italiana}
\inputminted[fontsize=\small,linenos,tabsize=2,breaklines]{text}{../output/kruskal_rede_italiana.txt}

\section{Rede USA}
\inputminted[fontsize=\small,linenos,tabsize=2,breaklines]{text}{../output/kruskal_rede_usa.txt}


\chapter{Algoritmo de Dijkstra}
\section{Resumo}
A lógica de implentação foi igual ao que foi apresentado em \cite{slidesaula}.

\section{Rede Italiana}
\foreach \x in {7,14,21}{
\subsection{De 1 a \x}
\inputminted[fontsize=\small,linenos,tabsize=2,breaklines]{text}{../output/dijkstra_rede_italiana_1_\x.txt}
}

\section{Rede USA}
\foreach \x in {10,20,30,40,50,60,70}{
\subsection{De 1 a \x}
\inputminted[fontsize=\small,linenos,tabsize=2,breaklines]{text}{../output/dijkstra_rede_usa_1_\x.txt}
}


\chapter{Algoritmo de Ford-Moore-Bellman}
\section{Resumo}
A lógica de implentação foi igual ao que foi apresentado em \cite{slidesaula}.

\section{Rede Italiana}
\foreach \x in {7,14,21}{
\subsection{De 1 a \x}
\inputminted[fontsize=\small,linenos,tabsize=2,breaklines]{text}{../output/ford_moore_bellman_rede_italiana_1_\x.txt}
}

\section{Rede USA}
\foreach \x in {10,20,30,40,50,60,70}{
\subsection{De 1 a \x}
\inputminted[fontsize=\small,linenos,tabsize=2,breaklines]{text}{../output/ford_moore_bellman_rede_usa_1_\x.txt}
}


\begin{thebibliography}{9}

\bibitem{python}
  Python Project,
  https://www.python.org/,
  Visitado em : \today
  
\bibitem{pythonwikipedia}
	Python - Wikipedia Article,
	https://pt.wikipedia.org/wiki/Python,
	Visitado em : \today
	
\bibitem{pythonperformance}
	Why is Python used for high-performance/scientific computing?,
	https://goo.gl/8aRXjN,
	Visitado em : \today

\bibitem{networkx}
  Networkx,
  https://networkx.github.io/,
  Visitado em : \today

\bibitem{networkxdocumentation}
  Networkx Lastest Documentation,
  https://networkx.readthedocs.io/en/stable/,
  Visitado em : \today
  
\bibitem{dfsalgorithm}
  Depth-first Algorithm,
  https://en.wikipedia.org/wiki/Depth-first\_search,
  Visitado em : \today

\bibitem{disjointsetalgorithm}
  Disjoint-set data structure,
  https://en.wikipedia.org/wiki/Disjoint-set\_data\_structure,
  Visitado em : \today  

\bibitem{slidesaula}
  Notas de Aula IA 881 - $2^{o}$ semestre de 2016,
  http://www.dt.fee.unicamp.br/~ricfow/IA881/ia881.htm,
  Visitado em : \today 
    
\end{thebibliography}


\begin{appendices}
\chapter{Prim}
\inputminted[fontsize=\footnotesize,linenos,breaklines]{python}{../prim.py}
\chapter{Kruskal}
\inputminted[fontsize=\footnotesize,linenos,breaklines]{python}{../kruskal.py}
\chapter{Dijkstra}
\inputminted[fontsize=\footnotesize,linenos,breaklines]{python}{../dijkstra.py}
\chapter{Ford-Moore-Bellman}
\inputminted[fontsize=\footnotesize,linenos,breaklines]{python}{../ford_moore_bellman.py}
\end{appendices}


\end{document}